\documentclass[11pt]{article}

% Common packages
\usepackage{amsmath}   % advanced math environments
\usepackage{amssymb}   % math symbols
\usepackage{amsthm}    % theorem/proof environments (optional, kept for later)
\usepackage{geometry}  % page margins
\usepackage{enumitem}  % better lists
\usepackage{graphicx}  % include images
\usepackage{hyperref}  % hyperlinks (keep this LAST)
\usepackage{titlesec}
\usepackage{multicol} % multiple columns
\usepackage{enumitem} % better list formatting
% Page setup
\geometry{margin=0.5in}
\setlist[itemize]{topsep=2pt,itemsep=2pt,parsep=0pt}

% Short labels
\newcommand{\thm}{\underline{\textbf{Thm.}} }
\newcommand{\defn}{\underline{\textbf{Def.}} }
\newcommand{\prop}{\underline{\textbf{Prop.}} }

% Title info
\title{Econ 4310 Notes}
\author{Michael Lee}

\begin{document}
\maketitle

\tableofcontents

\newpage

\section{Intro}
\begin{itemize}
    \item Roadmap:
    \begin{multicols}{3} % divide into 3 columns
        \begin{enumerate}[leftmargin=*]
            \item Intro
            \item Neoclassical benchmark
            \item Deviations from utility maximization
            \item Expected utility (EU) theory
            \item EU Deviations
            \item Dual process theory and bounded rationality
            \item Rational (in)attention and salience
            \item Satisficing
            \item Belief updating
            \item Overconfidence
            \item Game theory
            \item Fairness
            \item Trust and altruism
            \item Behavioral game theory
            \item Identity and norms
        \end{enumerate}
    \end{multicols}
    \item \defn Randomized experiment (RCT) - You randomly assign subjects to \(n\) groups with a control groups
    \item This is necessary to help separate correlation from causation
    \item Helpful because on average, the groups are the same. Thus, we can isolate the differentiating variable(s)
    \item \textbf{Experimental Language}
    \begin{itemize}
        \item \defn Treatment Group - Group exposed to the experimental condition
        \item \defn Control group - Baseline group
        \item \defn Sample - Group who makes up participants
        \item \defn Independent variable - Manipulated factor
        \item \defn Dependent variable - The measure of interest
        \item \defn Between-participant design - Each participant experiences only one condition
        \item \defn Within-participant design - Each participant experiences multiple conditions
        \item \defn Factorial design - Multiple factors are manipulated at once, often in all combinations
        \item \defn Internal validity - The degree to which the causal effect is valid inside the study
        \item \defn External validity - How generalizable the results are beyond the Sample
        \item \defn Confound - Uncontrollable factor(s) that may influence results
        \item \defn Incentive compatibility - Ensuring participants' best choice aligns with truthful or intended behavior
    \end{itemize}
    \item People might not be able to do RCT because economic systems are complex, there are cost/access problems, and ethical concerns
\end{itemize}

\section{What is Rational?}
\begin{itemize}
    \item \defn Rationality - Maximizing one's own self-interest (utility)
    \item Budget constraint and preferences \(\rightarrow\) optimal choice
    \item \defn Utility - represents a subject's preferences. When comparing the utility of options, comparisons are binary (not bundled), ordinal, and not comparable across subjects
    \begin{itemize}
        \item Utility can be anything (profit, social benefits, reproductive rights, combination of things, etc.)
    \end{itemize}
    \item \defn Preference Axioms
    \begin{itemize}
        \item Let \(X\) be a set of bundles
    \end{itemize}
    \begin{enumerate}
        \item \textbf{Reflexive} - \(\forall x \in X\), \(x \gtrsim x\) (i.e. each bundle is at least as preferred as itself)
        \item \textbf{Complete} - \(\forall x, y \in X\), \(x \gtrsim y\) or \(y \gtrsim x\) (or both (\(x \sim y\))) (i.e. every bundle can be compared to every other bundle)
        \item \textbf{Transitive} - \(\forall x,y,z \in X\), \(x \gtrsim y\) and \(y \gtrsim z \Rightarrow x \gtrsim z\)
    \end{enumerate}

    \item \defn Complete preference relation - \(X\) is reflexive, complete, and transitive \(\iff\) \(X\) is rationalizable
    \begin{itemize}
        \item The following statements are equivalent:
    \end{itemize}
    \[
        (xRy) = (x \gtrsim y) = (R = \{(x,y)\}) 
    \]
    \item \thm For finite \(X\), a binary relation, \(R\), on \(X\) has a utility representation \(\iff\) it is a complete preference relation
    \item Utility Function Assumptions
    \begin{enumerate}
        \item Continuous - There are no big jumps in changes
        \item Monotonic - More is better
        \item Convex - As the amount of smth increases, the marginal utlity gain decreases
    \end{enumerate}
\end{itemize}

\end{document}