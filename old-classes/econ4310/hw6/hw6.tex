\documentclass[11pt]{article}

% Common packages
\usepackage{amsmath}   % advanced math environments
\usepackage{amssymb}   % math symbols
\usepackage{amsthm}    % theorem/proof environments (optional, kept for later)
\usepackage{geometry}  % page margins
\usepackage{enumitem}  % better lists
\usepackage{graphicx}  % include images
\usepackage{titlesec}
\usepackage{hyperref}  % hyperlinks (keep this LAST)
\usepackage{tikz}
\usetikzlibrary{trees}
\usepackage{pgfplots}
\usetikzlibrary{arrows.meta,positioning}
\pgfplotsset{compat=1.18}

% Page setup
\geometry{margin=0.5in}
\setlist[itemize]{topsep=2pt,itemsep=2pt,parsep=0pt}

% Short labels
\newcommand{\thm}{\underline{\textbf{Thm.}} }
\newcommand{\defn}{\underline{\textbf{Def.}} }
\newcommand{\prop}{\underline{\textbf{Prop.}} }
\DeclareMathOperator*{\argmax}{arg\,max}


% Title info
\title{ECON 4310 HW5}
\author{Michael Lee}

\begin{document}
\maketitle

\begin{enumerate}[label=\textbf{(\arabic*)}, leftmargin=*]
    \item \textbf{Incentives in Experiemnts}
    
    \textit{Suppose you are designing an experiment and wish to understand a participant's valuation
    a 50/50 lottery on the outcomes \$10 and \$100. Describe a set of incentivized questions that
    you could ask in order to elicit the participant's valuation. What will the instructions to the
    participant be? How many questions will you ask? How will you pay them?}

    \textit{Hint: As you've seen in lab, the easiest way to elicit the valuation is to present a set of binary
    choices to make, where one item in each choice is the alternative in question, and you vary the
    other item. For example, if you offer a choice between (i) the alternative, and (ii) a certain
    dollar amount of \$50, and the participant chooses the \$50, you can infer that they value the
    alternative less than \$50.}

    \textbf{Set of incentivized questions.} I would ask the participant a set of questions where they have two options: (i) a 50/50 lottery on the outcomes \$10 and \$100, and (ii) a sure dollar amount, \(\$x_i\) , that varies between questions:
    \[
    \begin{array}{c|c|c}
        \text{Question} & \text{Option 1} & \text{Option 2} \\
        \hline
        Q1 & 50/50\text{ lottery on \$10\text{ and }\$100} & \$10 \\
        Q2 & 50/50\text{ lottery on \$10\text{ and }\$100} & \$20 \\
        Q3 & 50/50\text{ lottery on \$10\text{ and }\$100} & \$30 \\
        Q4 & 50/50\text{ lottery on \$10\text{ and }\$100} & \$40 \\
        Q5 & 50/50\text{ lottery on \$10\text{ and }\$100} & \$50 \\
        Q6 & 50/50\text{ lottery on \$10\text{ and }\$100} & \$60 \\
        Q7 & 50/50\text{ lottery on \$10\text{ and }\$100} & \$70 \\
        Q8 & 50/50\text{ lottery on \$10\text{ and }\$100} & \$80 \\
        Q9 & 50/50\text{ lottery on \$10\text{ and }\$100} & \$90 \\
        Q10 & 50/50\text{ lottery on \$10\text{ and }\$100} & \$100 \\
    \end{array}
    \]

    \textbf{Instrustions/Payment.} I would tell the participant they will make 10 choices. They can choose between a lottery that has a 50\% chance of winning \$10 and a 50\% chance of winning \$100, OR a sure dollar amount, \(\$x_i\). At the end of the experiment, one round will randomly be selected and they will get the payoff for that round.

    \item \textbf{Class lottery.} \textit{As you know, the points you accumulated in the lab sessions during the
    semester will be used to run a lottery for a \$100 gift card. Suppose that at the end of the
    semester, the total accumulated points by the entire class, including you, is 25,000. Suppose
    your total is 600, and suppose I were to offer you a bonus of 70 points at the end of the
    semester. How much, in dollar terms, is the bonus worth?}

    Assuming the 70 point bonus is given to me and only me, the new total points is 25,700. Thus, the value per point is:
    \[
    a = \frac{\$100}{25,070} = \approx 0.00398 \$/point
    \]
    Thus, the value of the 70 point bonus is:
    \[
    70 \cdot a \approx \$0.28
    \]

    \newpage
    \item \textbf{QRE.}
    
    \[
    p(x_j) = \frac{\exp(\lambda u (x_j))}{\sum_{x_k \in A}\exp(\lambda u (x_k))}
    \quad (\text{where }\lambda = 1.5)
    \]

    \begin{enumerate}[label=\textbf{(\alph*)}, leftmargin=*]
        \item \textbf{ }
        \[
        p_2^4(T)
        = \frac{\exp(1.5 \cdot 3.2)}{\exp(1.5 \cdot 3.2) + \exp(1.5 \cdot 1.6)}
        \approx 0.917
        \]
        
        \item \textbf{ }
        \[
            u_1^3(P)
            = p_2^4(T) \cdot 0.80 + (1 - p_2^4(T)) \cdot 6.40
            = 0.917 \cdot 0.80 + (1 - 0.917) \cdot 6.40
            \approx 1.266
        \]

        \[
            p_1^3(T)
            = \frac{\exp(1.5 \cdot 1.60)}{\exp(1.5 \cdot 1.60) + \exp(1.5 \cdot 1.266)}
            \approx 0.623
        \]
        
        \item \textbf{ }
        \[
            u_2^2(P)
            = p_1^3(T) \cdot 0.40 + (1 - p_1^3(T))(p_2^4(T) \cdot 3.20 + (1 - p_2^4(T)) \cdot 1.60)
        \]
        \[
            = 0.623 \cdot 0.40 + (1 - 0.623) \cdot (0.917 \cdot 3.20 + (1 - 0.917) \cdot 1.60)
            \approx 1.406
        \]
    
        \[
            p_2^2(T)
            = \frac{\exp(1.5 \cdot 0.80)}{\exp(1.5 \cdot 0.80) + \exp(1.5 \cdot 1.406)}
            \approx 0.287
        \]
        
        \item \textbf{ }
        
        \[
        u_1^1(P)
        = p_2^2(T) \cdot 0.20 + (1 - p_2^2(T))(p_1^3(T) \cdot 1.60 + (1 - p_1^3(T))(p_2^4(T) \cdot 0.80 + (1 - p_2^4(T)) \cdot 6.40))
        \]
        \[
        = 0.287 \cdot 0.20 + (1 - 0.287)(0.623 \cdot 1.60 + (1 - 0.623)(0.917 \cdot 0.8 + (1 - 0.917) \cdot 6.40))
        \approx 1.108
        \]

        \[
            p_1^2(T)
            = \frac{\exp(1.5 \cdot 0.40)}{\exp(1.5 \cdot 0.40) + \exp(1.5 \cdot 1.108)}
            \approx 0.257
        \]

    \end{enumerate}

    \item \textbf{k-level Thinking}
    
    \begin{enumerate}[label=\textbf{(\alph*)}, leftmargin=*]
        \item \textbf{ }
        At level 0, players play uniformly, so \(p^t(0) = 0.5\) at all decision nodes.
        
        \item \textbf{ }
        At level 1, we assume our opponent is precisely type 0, so their probability of playing T is \(p^t(1) = 0.5\) at all decision nodes. We will start at Node 4 and go backwards. Let the utility of playing \(a_i\) in round \(j\) be \(u_j(a_i)\) and the utility of any given round be \(u_j\) \\

        \textbf{Node 4:}
        \[
            u_4(T) = 3.20 > u_4(P) = 1.60 \quad \Rightarrow \quad \text{Play T}
        \]
        Hence,
        \[
        p^4(1) = 1
        \]

        \textbf{Node 3:}
        \[
        u_3(T)
        = 1.60
        < u_3(P)
        = p_4(T) \cdot u_4(T) + p_4(P) \cdot u_4(P)
        = 0.5 \cdot 0.80 + 0.5 \cdot 6.40
        = 3.60
        \quad \Rightarrow \quad \text{Play P}
        \]
        Hence,
        \[
        p^3(1) = 0
        \]

        \textbf{Node 2:}
        \[
        u_2(T)
        = 0.80
        < u_2(P)
        = p_3(T) \cdot u_3(T) + p_3(P) \cdot u_3(P)
        = 0.5 \cdot 0.40 + 0.5 \cdot 3.20
        = 1.80
        \quad \Rightarrow \quad \text{Play P}
        \]
        Hence,
        \[
        p^2(1) = 0
        \]
        
        \textbf{Node 1:}
        \[
        u_1(T)
        = 0.40
        < u_1(P)
        = p_2(T) \cdot u_2(T) + p_2(P) \cdot u_2(P)
        = 0.5 \cdot 0.20 + 0.5 \cdot 1.80
        = 1.00
        \quad \Rightarrow \quad \text{Play P}
        \]
        Hence,
        \[
        p^1(1) = 0
        \]

        \item \textbf{ }
        At level 2, we assume our opponent is precisely type 1, so they will play \(T\) in Node 4 and \(P\) in all other nodes. Again, we will start at Node 4 and go backwards.

        \textbf{Node 4:}
        \[
            u_4(T) = 3.20 > u_4(P) = 1.60 \quad \Rightarrow \quad \text{Play T}
        \]
        Hence,
        \[
        p^4(2) = 1
        \]

        \textbf{Node 3:}
        \[
        u_3(T) = 1.60 > u_3(P) = p^4(1) \cdot u_4(T) + (1 - p^4(1)) \cdot u_4(P) = 1 \cdot 0.80 = 0.80 \quad \Rightarrow \quad \text{Play T}
        \]
        Hence,
        \[
        p^3(2) = 1
        \]

        \textbf{Node 2:}
        \[
        u_2(T) = 0.80 < u_2(P) = p^3(1) \cdot 0.4 + (1 - p^3(1)) \cdot u_3(P) = 0 \cdot 0.40 + 1 \cdot 3.20 = 3.20 \quad \Rightarrow \quad \text{Play P}
        \]
        Hence,
        \[
        p^2(2) = 0
        \]

        \textbf{Node 1:}
        \[
        u_1(T) = 0.40 < u_1(P) = p^2(1) \cdot 0.20 + (1 - p^2(1)) \cdot u_3(T) = 0 \cdot 0.20 + 1 \cdot 1.60 = 1.60 \quad \Rightarrow \quad \text{Play P}
        \]
        Hence,
        \[
        p^1(2) = 0
        \]

    \end{enumerate}
        
\end{enumerate}

\end{document}  