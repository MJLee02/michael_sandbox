\documentclass[11pt]{article}

% Common packages
\usepackage{amsmath}   % advanced math environments
\usepackage{amssymb}   % math symbols
\usepackage{amsthm}    % theorem/proof environments (optional, kept for later)
\usepackage{geometry}  % page margins
\usepackage{enumitem}  % better lists
\usepackage{graphicx}  % include images
\usepackage{titlesec}

\usepackage{hyperref}  % hyperlinks (keep this LAST)

% Page setup
\geometry{margin=0.5in}
\setlist[itemize]{topsep=2pt,itemsep=2pt,parsep=0pt}

% Short labels
\newcommand{\thm}{\underline{\textbf{Thm.}} }
\newcommand{\defn}{\underline{\textbf{Def.}} }
\newcommand{\prop}{\underline{\textbf{Prop.}} }

% Title info
\title{ESE 4170 HW1}
\author{Michael Lee}

\begin{document}

\section*{Problem 2}
Consider the following matrix:
\[
    A = \begin{bmatrix}
        5 & 4 \\
        4 & 5
    \end{bmatrix}
\]
\begin{enumerate}[label=(\alph*), leftmargin=*]
    \item Find the eigenvalues and corresponding eigenvectors of matrix \(A\)
    \[
        |A - \lambda I|
        =
        \left|
        \begin{bmatrix}
            5 - \lambda & 4 \\
            4 & 5 - \lambda
        \end{bmatrix}
        \right|
        = (5 - \lambda)^2 - 4^2
        = 25 - 10\lambda + \lambda^2 - 16
        = \lambda^2 - 10\lambda + 9
        = (\lambda - 1)(\lambda - 9)
        = 0
    \]
    \[
    \Rightarrow \lambda_1 = 1 \text{ and } \lambda_2 = 9
    \]
    Let the eigenvector of \(\lambda_i\) be \(v_{i}\). \\

    First, we will find \(v_1\):
    \[
        (A - \lambda_1 I)v_1
        =
        \begin{bmatrix}
            4 & 4 \\
            4 & 4
        \end{bmatrix} v_1
        = \vec{0}
    \]
    Via Gaussian elimination,
    \[
        \left[
            \begin{array}{c c | c}
                4 & 4 & 0 \\
                4 & 4 & 0
            \end{array}
        \right]
        \Rightarrow
        \left[
            \begin{array}{c c | c}
                1 & 1 & 0 \\
                0 & 0 & 0
            \end{array}
        \right]
        \Rightarrow
        v_1 = 
        \begin{bmatrix}
            a \\
            -a
        \end{bmatrix}
        \quad \text{where} \quad a \in \mathcal{R}
    \]
    Then, we will find \(v_2\):
    \[
        (A - \lambda_1 I)v_2
        =
        \begin{bmatrix}
            -4 & 4 \\
            4 & -4
        \end{bmatrix} v_2
        = \vec{0}
    \]
    Via Gaussian elimination,
    \[
    \left[
        \begin{array}{c c | c}
            -4 & 4 & 0 \\
            4 & -4 & 0 
        \end{array}
    \right]
    \Rightarrow
    \left[
        \begin{array}{c c | c}
            -1 & 1 & 0 \\
            0 & 0 & 0 
        \end{array}
    \right]
    \Rightarrow
    v_2 =
    \begin{bmatrix}
        b \\
        b
    \end{bmatrix}
    \quad \text{where} \quad b \in \mathcal{R}
    \]
    Thus,
    \[
    \boxed{
        \lambda_1 = 1,
        \quad \lambda_2 = 9,
        \quad v_1
        =
        \begin{bmatrix}
            a \\
            -a
        \end{bmatrix},
        \quad v_2
        =
        \begin{bmatrix}
            b \\
            b
        \end{bmatrix}
        \quad \text{where} \quad
        (a, b \in \mathcal{R})
    }
    \]

    \item Find the eigendecomposition of matrix \(A = Q \Lambda Q^T\) using the results from a)
    
    Given \(\lambda_1, \lambda_2\), let \(\Lambda\) be
    \[
        \Lambda = 
        \begin{bmatrix}
            1 & 0 \\
            0 & 9
        \end{bmatrix}
        \quad \text{and so} \quad
        Q =
        \frac{\sqrt{2}}{2}
        \begin{bmatrix}
            1 & 1 \\
            -1 & 1
        \end{bmatrix}
    \]
    Verifying our solution,
    \[
        A = Q \Lambda Q^T
        = \frac{\sqrt{2}}{2}
        \begin{bmatrix}
            1 & 1 \\
            -1 & 1
        \end{bmatrix}
        \begin{bmatrix}
            1 & 0 \\
            0 & 9
        \end{bmatrix}
        \frac{\sqrt{2}}{2}
        \begin{bmatrix}
            1 & -1 \\
            1 & 1
        \end{bmatrix}
        =
        \frac{1}{2}
        \begin{bmatrix}
            1 & 9 \\
            -1 & 9
        \end{bmatrix}
        \begin{bmatrix}
            1 & -1 \\
            1 & 1
        \end{bmatrix}
        \frac{1}{2}
        \begin{bmatrix}
            10 & 8 \\
            8 & 10
        \end{bmatrix}
        =
        \begin{bmatrix}
            5 & 4 \\
            4 & 5
        \end{bmatrix}
    \]

    \item What is the definiteness of the matrix?
    
    \(A\) is positive-definite as \(\lambda_1, \lambda_2 > 0\).

\end{enumerate}

\newpage

\section*{Problem 4}
\begin{proof}
    Use the definition of a convex function to show that affine functions \(f(x) = a^T x + b\) are convex, where \(a, x \in \mathcal{R}^n\) and \(b \in \mathcal{R}\). \\

    \noindent Let \(x_1, x_2 \in \mathcal{R}^n\). Recall that a function \(f: \mathcal{R}^n \to \mathcal{R}\) is convex if
    \[
        f\big(t x_1 + (1-t) x_2\big) \leq t\,f(x_1) + (1-t)\,f(x_2)
        \quad \text{for all } x_1, x_2 \text{ and } t \in [0,1].
    \]
    Looking at \(f\),
    \[
        f\big(t x_1 + (1-t) x _2\big)
        = a^T\big(t x_1 + x_2 - t x_2\big) + b
        \geq t \, f(x_1) + (1-t)\,f(x_2)
        = t \big( a^T x_1 + b \big)
    \]
    \[
    a^T \big( t x_1 + x_2 - t x_2 \big) \geq t\big( a^T x_1 + b \big) + (1-t) \big( a^T x_2 + b \big)
    \]
    \[
        t a^T x_1 + a^T x_2 - t a^T x_2 \geq ta^T x_1 + tb + a^T x_2 + b - t a^T x_2 - tb
    \]
    \[
        0 \geq 0
    \]
    The inequality holds, and so affine functions \(f(x) = a^T x + b\) are convex.
\end{proof}

\section*{Problem 5}
Consider the following objective function:
\[
    f(x,y) = x^2 + 2y^2 - 2xy
\]
\begin{enumerate}[label=(\alph*), leftmargin=*]
    \item Compute the Hessian matrix \(H\) of function \(f\), and find the definiteness of matrix \(H\)
    First, we will compute the Hessian matrix of \(f\), \(H\):
    \[
        f
        \Rightarrow
        \nabla f
        =
        \begin{bmatrix}
            2x - 2y \\
            4y - 2x
        \end{bmatrix}
        \Rightarrow
        H
        = 
        \begin{bmatrix}
            2 & -2 \\
            -2 & 4
        \end{bmatrix}
    \]
    Now we will determine the definiteness of \(H\) by finding its eigenvalues:
    \[
    \left|\big( H - \lambda I \big)\right|
    = \left|\begin{bmatrix}
        2 - \lambda & -2 \\
        -2 & 4 - \lambda
    \end{bmatrix}\right|
    = (2 - \lambda)(4 - \lambda) - (-2)^2
    = \lambda^2 - 6 \lambda + 4 = 0
    \]
    \[
    \lambda
    = \frac{6 \pm \sqrt{(-6)^2 - 4(1)(4)}}{2(1)}
    = 3 \pm \sqrt{5} > 0
    \]
    Thus, \(H\) is positive-definite

    \item Use calculus method to find the minimum value of \(f(x, y)\)
    \[
        \nabla f = \vec{0} \Rightarrow (x, y) = (0, 0)
    \]
    \[
        f_{xx} = 2 \quad \text{and} \quad |H| = 8 - (-2)^2 = 4 \Rightarrow (0,0) \text{ is the global minimum of } f
    \]
\end{enumerate}

\end{document}